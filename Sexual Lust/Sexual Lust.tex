\documentclass[a4paper, parskip=full, 12pt]{article}
\usepackage{parskip}
\usepackage{xparse}
\setcounter{secnumdepth}{4}
\setcounter{section}{0}
\begin{document}
\section*{Sexual Lust}
\subsection{Remarks}
Every Christian in the church has at one point heard that it is wrong to lust. Very few are proud of lust. The church perpetuates this association by always making lust mutually exclusive with something that we must accept as good. For instance, we often hear of the dichotomy of love and lust, with love being associated with whatever passion is good and lust being associated with whatever passion is bad. Love is viewed as genuine while lust is viewed as shallow. I once heard someone blurt some nonsense, saying that "love is a focus on the person being loved whereas lust is a focus on the person with the lust". I might as well have heard that "persuading is about the person hearing my argument whereas manipulating is about the person pulling the strings". Just becuase you draw emphasis on certain parts of an action doesn't make those parts more important to the vern (more like nominalization) being used. Love requires a lover as much as someone to tbe loved in the same way that lust requires an object of lust as much as someone with lust. Back to the main point, very rarely do we believe that we can both "love" and "lust" after a girl and if someone did say that, many Christians would feel that that person was disingenuous. It is always one or the other. This is just one of the many ways that we have learned to associate lust with "something bad" even before we can discuss what is being lusted after. \\
\break
In the Bible, the Green noun which is often translated as "lust" simply means "desire". This word, "epithymia", means a desire for anything, whether desire for holiness or desire for evil. However, in order to differentiate between these distinct usages of "epithymia", the translaters of the Bible made value judgments to decide what English word would fit the context of the verse. I explain this to show that when stripped of its negative connotation, lust isn't wrong at all. Sure, lust is wrong when you're lusting after your neighbor's wife, but no one should shudder when they hear of someone with "lust". \\
\break
Consequently, the first step you need to take immediately is to detach the word "lust" with any negative feelings in your stomach that result from saying it. If you decide to continue to focus on the English translation and neglect the original Greek words, you become victim to how others use the word "lust". For instance, some people may say that you can either "love God" or "lust after a woman". No one says you can either "lust after God" or "love a woman". The problem is that we let the words control how we feel about the morality of the statement. Rather than taking responsibility for interpreting the passage, we're letting the words on the page interpret themselves for us by hijacking our distaste for the word "lust" as used in colloquial language. \\
\break
The second step you must take is to realize that "lust" is a nominalization of the verb "lusting". "Lust" doesn't exist because it doesn't refer to anything. In order to unmask what a person means by "lust", make the person use the word "lusting". After all, they must happen together. No one can have "lust" if that person is not "lusting" after something. \\
\break
A critic may object that the connotation for the word "lust" exists for a reason and that we should honor the passage by using the words "desire", "lust", and "passion" appropriately. This critic may suppose that we should simply use "lust" when we're referring to something sinful and use other words when we're referring to something good. First of all, the connotation behind the word "lust" does exist for a reason, but this reason is crappy as I explained just now. Secondly, simply because the masses place high emotional stock in the connotations of the words we use doesn't mean that we have to follow and allow the connotations of words to affect us. Part of unplugging is to realize how people use wordds to influence others. For this reason, every person should learn to analyze the content of words and not allow the specific word choices the author used to influence their feelings about a position. Lastly, we have to cut off the negative connotations we internally have for the word "lust" since this negative association is one of the immediate reasons why people believe it is wrong to lust after a woman. In this way, dissociating the word lust with its common negative context helps us to look at the subject of lust more objectively. \\
\subsection{Textual Analysis}

Now, let's examine Matthew 5:27-29 taken from the Sermon on the Mount.

“You have heard that it was said, ‘You shall not commit adultery.’ But I say to you that everyone who looks at a woman with lustful intent has already committed adultery with her in his heart. If your right eye causes you to sin, tear it out and throw it away. For it is better that you lose one of your members than that your whole body be thrown into hell." - Matthew 5:27-29 (ESV)

“You have heard that it was said, ‘You shall not commit adultery.’But I tell you that anyone who looks at a woman lustfully has already committed adultery with her in his heart. If your right eye causes you to stumble, gouge it out and throw it away. It is better for you to lose one part of your body than for your whole body to be thrown into hell." - Matthew 5:27-29 (ESV)

"Ye have heard that it was said by them of old time, Thou shalt not commit adultery: But I say unto you, That whosoever looketh on a woman to lust after her hath committed adultery with her already in his heart. And if thy right eye offend thee, pluck it out, and cast it from thee: for it is profitable for thee that one of thy members should perish, and not that thy whole body should be cast into hell." - Matthew 5:27-29 (KJV)

I am no expert on ancient Greek. I am no Biblical scholar at all. Yet, to anyone reading the verse, it should be clear that Jesus is talking about adultery, which is a man having sex with another man's wife (or a woman having sex with another woman's husband). For this reason, the word "woman" in the passage should be viewed as "wife". This passage indicates that the man already commits adultery in his heart by having the intention to commit the deed before he actually carries out adultery with a married woman. This passage is not for guys and girls who are not married.

This is the general gist you need to know to prepare yourself against someone who's still plugged into the normal church culture.
\subsection*{Arguments}
\subsubsection*{Scriptural Evidence}
\subsubsubsection*{Matthew 5:28}
  \item Matthew 5:28 demonstrates that lust is wrong.
  \begin{itemize}
    \item Description of argument: \textit{"Matthew 5:28 clearly shows that lust is wrong."}
    \item I respond: "Taking an appropriate look of the entire Sermon of the Mount and Jewish views of adultery, it's clear that Jesus is talking about lust within the context of a married relationship."
  \end{itemize}
  \item The future wife argument
  \begin{itemize}
    \item Description of argument: \textit{"By lusting after a woman, you're committing adultery against your future wife. Would your future wife want you to look at the woman you're looking at now?"}
    \item I respond: "First off, you don't have a future wife just like you don't have a future job or a future \$1,000,000 salary. Imagine that you're single and that you die now. Then it would appear that you don't have a future wife. Adding "future" in front of a word doesn't make it real. And since you can't commit adultery with someone that isn't real, you can't sin through this verse.
    \item I respond: "If the future wife argument was valid, you wouldn't be able to know if the person you're lusting after is indeed your future wife, so you can never know if you're sinning. I suppose people who use the future wife argument imagine that right after your wedding day, all your past lusts of other woman suddenly come and bite your ass to fulfill Matthew 5:28."
    \item I respond: "You wouldn't be able to marry anyone (out of your own free will) if you were trying not to sin under the future wife argument. Unless you plan to be arranged into marriage, you will most likely marry your significant other because you have an intense passion for her to be your wife. With the way most people word the future wife argument, this could count as an act of coveting since you are wanting someone to be yourself when she does not belong to you."
  \end{itemize}
  \item The sin against oneself argument
  \begin{itemize}
    \item Description of argument: \textit{"By lusting after her, you're sinning against your own body. It doesn't matter if you are married or not."}
    \item I respond: "Jesus mentions that he who sins sexually sins against his own body. However, we can't talk about the implication of sexual sin before we establish what should be considered as sexual sin. Refer to my response to Matthew 5:28 for why lusting as a single man should not be considered sexual sin."
  \end{itemize}
  \item The idolatry argument
  \begin{itemize}
    \item Description of argument: \textit{"It doesn't matter what your interpretation of Matthew 5:28 is, the larger context of the Bible shows that lusting after a girl is idolatry. You're valuing the creation over the creator."}
    \item I respond: "In most Christian discussions, we seem to be battling over who has the bigger context. You often here people claiming that they have the biggest context and that they're taking into account more things. All of this aside, this argument fails because it assumes that the person with lust is valuing the creation over the creator. It's perfectly reasonable if not more likely that a Christian feels in awe of a beautiful girl just like he feels in awe of a beautiful sunset. In both cases, the person is grateful for the creator as much as he is grateful for the girl in his life. Simply because he is passionate about having a sexual relationship with her does not mean he is an idolator."
  \end{itemize}
  \item The disingenuous argument
  \begin{itemize}
    \item Description of argument: \textit{"You're just saying this because you want to give yourself an excuse to lust as a single man."}
    \item I respond: "This response doubts the integrity of the person who's proposing that lust isn't sinful instead of dealing with the argument."
  \end{itemize}
\end{enumerate}
\subsection{Mindsets of the Interloculator}
Now let's go over the mindsets you'll encounter from those plugged into the church.

But let's think through the mindset of someone who's \textbf{unplugged}.
\end{document}
