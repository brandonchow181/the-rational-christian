\documentclass[a4paper, parskip=full, 12pt]{article}
\usepackage{parskip}
\usepackage{xparse}
\usepackage{titlesec}
\setcounter{section}{0}
\begin{document}
\section{On Lust (Three Articles)}
We will divide our discussion of lust into three parts: (1) Of the definition of lust; (2) Of the sinfulness of sexual lust; (3) Of whether Christians should accept sexual lust if they wish to follow Christ.
\subsection{Whether lust is a desire?}
\textbf{Objection 1} It seems that lust describes a sinful sensual appetite because the word "lust" in the Bible is generally used to represent activities that are morally wrong. \\
\textbf{Objection 2} It seems that lust represents sinful passions because people in our current society apply a negative connotation to the word "lust". \\
\textbf{On the contrary,} the Greek noun for lust "epithymia" translates as "desire". It does not specify whether the thing being desired is sinful or not. \\
\textbf{I answer that,} when people see the word "lust" in the Bible, all they're seeing is a word that means "desire". Without putting the word "epithymia" in any context, the action of lusting is not wrong or right in and of itself. Anyone who continues to believes that lust by definition is wrong should refer to \textit{slave to langauge}. \\
\textbf{Reply to Objection 1} People believe that lust is biblically wrong because the translators of the Bible translated the word "epithymia" as "lust" when the Bible was judging a certain behavior as morally wrong. However, we can use the word for lust "epithymia" to describe a lust for God. Therefore, lust does not describe a sinful sensual appetite. \\
\textbf{Reply to Objection 2} People believe that lust is wrong because people associate the word "lust" with sinful desires such as "lusting after your neighbor's wife" or "lusting for revenge". Stripped of these negative connotations, there is no reason for anyone to view lust as wrong. For this reason, it is good practice to differentiate between the meaning behind the word "lust" and our associations with the word "lust" so that we do not cloud our judgment with our own psychological biases. \\
\subsection{Whether it is sinful to lust after a man or woman?}
\textbf{Objection 1} According to Matthew 5:27-29, Jesus states that anyone who "looks at a woman lustfully" already commits the sinful act of adultery with her. Therefore, it is sinful to lust after a man or woman. \\
\textbf{Objection 2} \\
\textbf{On the contrary,} in Matthew 5:27-29, Jesus is talking about adultery ie. a violation of a relationship between a married couple.\\
\textbf{I answer that,} \\
\textbf{Reply to Objection 1} Now, let's examine Matthew 5:27-29 taken from the Sermon on the Mount.
“You have heard that it was said, ‘You shall not commit adultery.’But I tell you that anyone who looks at a woman lustfully has already committed adultery with her in his heart. If your right eye causes you to stumble, gouge it out and throw it away. It is better for you to lose one part of your body than for your whole body to be thrown into hell." - Matthew 5:27-29 (ESV)

"Ye have heard that it was said by them of old time, Thou shalt not commit adultery: But I say unto you, That whosoever looketh on a woman to lust after her hath committed adultery with her already in his heart. And if thy right eye offend thee, pluck it out, and cast it from thee: for it is profitable for thee that one of thy members should perish, and not that thy whole body should be cast into hell." - Matthew 5:27-29 (KJV)

I am no expert on ancient Greek. I am no Biblical scholar at all. Yet, to anyone reading the verse, it should be clear that Jesus is talking about adultery, which is a man having sex with another man's wife (or a woman having sex with another woman's husband). For this reason, the word "woman" in the passage should be viewed as "wife". This passage indicates that the man already commits adultery in his heart by having the intention to commit the deed before he actually carries out adultery with a married woman. This passage is not for guys and girls who are not married. \\

\subsection{Whether single Christians should accept sexual lust if they wish to follow Christ?}
\textbf{Objection 1 (The Sin-against-self Argument)} It seems that lusting after a man or woman is an act of sin against one's own body. By lusting after her, you're sinning against your own body as all sexual sin is according to Jesus. \\
\textbf{Objection 2 (The Future Wife Argument)} It seems that lusting after a man or woman is an act of adultery against one's future spouse. In order to follow Christ, we must rid ourselves of any sort of adultery. For this reason, it is wrong to lust as a Christian. \\
\textbf{Objection 3 (The Idolatry Argument)} It seems that it doesn't matter what your interpretation of Matthew 5:28 is, the larger context of the Bible shows that lusting after a girl is idolatry. You're valuing the creation over the creator. \\
\textbf{Objection 4 (The Stumbling Block Argument)} We should not openly say that it is fine to lust after a man or woman because we will then cause other Christians to stumble. Therefore, whether or not sexual lust for a single person is sinful or not, we should discourage sexual lust as much as possible for the sake of preserving the Christian community. \\
\textbf{On the contrary,} for many people, sexual lust is necessary for a Christian man or woman to form a strong bond with his or partner. Most relationships begin with some sort of lust (intense passion) of a sexual nature. Whether or not this desire is specifically penetration or kissing or to be intimate with this other person, such a passion is lust. \\
\textbf{I answer that,} the post-rationaliations that people make to show that lust is wrong.
\textbf{Reply to Objection 1 (The Sin-against-self Argument)} Jesus mentions that he who sins sexually sins against his own body. However, we can't talk about the implication of sexual sin before we establish what should be considered as sexual sin. Refer to my response to Matthew 5:28 for why lusting as a single man should not be considered sexual sin." \\
\textbf{Reply to Objection 2 (The Future Wife Argument)} First off, you don't have a future wife just like you don't have a future job or a future \$1,000,000 salary. Imagine that you're single and that you die now. Then it would appear that you don't have a future wife. Adding "future" in front of a word doesn't make it real. And since you can't commit adultery with someone that isn't real, you can't sin through this verse. If the future wife argument was valid, you wouldn't be able to know if the person you're lusting after is indeed your future wife, so you can never know if you're sinning. I suppose people who use the future wife argument imagine that right after your wedding day, all your past lusts of other woman suddenly come and bite your ass to fulfill Matthew 5:28.
You wouldn't be able to marry anyone (out of your own free will) if you were trying not to sin under the future wife argument. Unless you plan to be arranged into marriage, you will most likely marry your significant other because you have an intense passion for her to be your wife. With the way most people word the future wife argument, this could count as an act of coveting since you are wanting someone to be yourself when she does not belong to you. \\
\textbf{Reply to Objection 3 (The Idolatry Argument)} This argument fails because it assumes that the person with lust is valuing the creation over the creator. It's perfectly reasonable if not more likely that a Christian feels in awe of a beautiful girl just like he feels in awe of a beautiful sunset. In both cases, the person is grateful for the creator as much as he is grateful for the girl in his life. Simply because he is passionate about having a sexual relationship with her does not mean he is an idolator. \\
\textbf{Reply to Objection 4 (The Stumbling Block Argument)} Discussing whether or not sexual lust is sinful is not the same as showing a picture of a naked girl to another Christian. Refer to \textit{the stumbling block argument}. \\
\end{document}
