\documentclass[a4paper, parskip=full, 12pt]{article}
\usepackage{parskip}
\usepackage{xparse}
\usepackage{titlesec}
\setcounter{section}{0}
\begin{document}
\section{On Swearing (Three Articles)}
Swearing is very likely the most irrational stigma within the Christian community. In the context I'm using the word, I don't mean "swearing" as "to make an oath" as it is used in the Bible. Funny enough, the Bible actually says that it is wrong "to make an oath", which I wholeheartedly agree with. However, when I use the word "swearing", I'm referring to the act of saying certain words that have been deemed too vulgar to say out loud. For most people, the word "cussing" most likely fits what I mean by "swearing" but I will continue to use the word "swearing" because that is the word I am accustomed to using to describe the use of vulgar langauge. \\
\break
Swearing is also an interesting stigma because it depends on the common language that a group of people deem vulgar. The moment that a word is no longer deemed vulgar, people no longer view that word as a "swear word" and freely use the word without fear of judgment from others. Notice how Sswear words often go in and out of the swearing category over time. You can find a traditional swear word like "crap" in a Christian conversation near you or even out of the mouth of a church pastor. Understanding the absurdity that a sin could be dependent on the whims of a mass group of people should already make a Christian begin to question the sinfulness of swearing. \\

\subsection{Whether swear words exist?}
\textbf{Objection 1} It seems that swear words exist.
\textbf{On the contrary,} swear words are arbitrary labels.
\textbf{I answer that,} swear words do not exist.
\textbf{Reply to Objection 1}

\subsection{Whether it is sinful to swear?}
\textbf{Objection 1} It seems that it is sinful to swear. Proverbs 6:12-15 says that "A worthless person, a wicked man, goes about with crooked speech, winks with his eyes, signals with his feet, points with his finger, with perverted heart devises evil, continually sowing discord; therefore calamity will come upon him suddenly; in a moment he will be broken beyond healing." (ESV)
\textbf{Objection 2} It seems that it is sinful to swear. Ecclesiastes 10:13 says that "The beginning of the words of his mouth is foolishness, and the end of his talk is evil madness." (ESV)"
\textbf{Objection 3} It seems that it is sinful to swear. Mark 7:20 says that "And he said, 'What comes out of a person is what defiles him. For from within, out of the heart of man, come evil thoughts, sexual immorality, theft, murder, adultery, coveting, wickedness, deceit, sensuality, envy, slander, pride, foolishness. All these evil things come from within, and they defile a person.'" (ESV)
\textbf{Objection 4} It seems that it is sinful to swear. Galatians 5:22 says that "But the fruit of the Spirit is love, joy, peace, patience, kindness, goodness, faithfulness, gentleness, self-control; against such things there is no law." (ESV)
\textbf{Objection 5} Ephesians 5:4 says that "Let there be no filthiness nor foolish talk nor crude joking, which are out of place, but instead let there be thanksgiving." (ESV)
\textbf{On the contrary,}
\textbf{I answer that,}
\textbf{Reply to Objection 1}
\textbf{Reply to Objection 2}
\textbf{Reply to Objection 3}
\textbf{Reply to Objection 4}
\textbf{Reply to Objection 5}

\subsection{Whether Christians should swear if they wish to follow Christ?}
\subsection{Taking A Look At Particular Swear Words}
\paragraph{Damn}
\paragraph{Fuck}
\paragraph{Shit}
\paragraph{Genitals (Dick, Pussy, Cunt)}
\end{document}




  \item Romans 12:16
  \begin{itemize}
    \item Passage: "Live in harmony with one another. Do not be haughty, but associate with the lowly. Never be wise in your own sight." (ESV)
  \end{itemize}


  \item 2 Corinthians 5:20
  \begin{itemize}
    \item Passage: "Therefore, we are ambassadors for Christ, God making his appeal through us. We implore you on behalf of Christ, be reconciled to God." (ESV)
  \end{itemize}
