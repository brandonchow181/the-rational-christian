\documentclass[a4paper, 12pt]{article}
\usepackage[T1]{fontenc}
\usepackage[utf8]{inputenc}
\usepackage{babel}
\usepackage[raggedright]{titlesec}
\usepackage{blindtext}
\titleformat{\paragraph}[hang]{\normalfont\normalsize\bfseries}{\theparagraph}{1em}{}
\titlespacing*{\paragraph}{0pt}{3.25ex plus 1ex minus .2ex}{0.5em}
\setcounter{section}{0}
\begin{document}
\section*{Nominalizations}
\paragraph{Definition Of A Nominalization}
According to the Oxford Dictionary, to nominalize a word is \textit{to form (a noun) from a verb or adjective, e.g. output, truth, from put out, true}. We say that a nominalization is a word that has been nominalized. For example, the noun ``communication'' is a nominalization since it is formed from the verb ``to communicate'' and the noun ``love'' is a nominalization since it is formed from the verb ``to love''. However, the noun ``book'' and the proper noun ``John'' are not nominalizations since they are not formed from any verbs/adjectives. Once you understand what a nominalization is, notice that ``nominalization'' is itself a nominalization since the word "nominalization" is formed from the verb ``to nominalize''. In fact, the entry for ``nominalization'' on Oxford Dictionaries only says to \textit{see nominalize}, indicating that the word is formulated from its verb counterpart. \\
\paragraph{Nominalizations Cannot Be Understood Without The Verb/Adjective They Are Formulated From}
Notice that not all nouns are nominalizations, but only those nouns that can only be made sense through the verb/adjective that the nominalization is based upon. For example, the word ``communication'' is a nominalization because the word ``communication'' has meaning only if it describes someone communicating to someone else. A critic may claim that nominalizations do not always have this feature by indicating that the word ``communication'' can be explained without the verb ``communicating''. Although the Oxford Dictionary does define ``communication'' as \textit{the imparting or exchanging of information by speaking, writing, or using some other medium}, notice that the definition of the word ``communication'' itself contains more nominalizations such as ``imparting'' and ``exchanging'', both of which can only be understood through something being imparted or something being exchanged. Since every nominalization can only be explained through its verb/adjective counterpart or another nominalization, all nominalizations reduce to verbs/adjectivess when their meanings are examined. \\
\paragraph{The Function Of A Nominalization}
Nominalizations only make sense (ie. their meaning can be understood) when they are used as substitutes for verbs/adjectives. For example, the sentence "I'm looking for the truth" is a substitute for the sentence "I'm looking for what is true". The nominalization allows a sentence to convey more meaning than it could without it. For this reason, if you see a sentence with many big words such as the following sentence taken G.K. Chesterton's work, ``the social utility of the indeterminate sentence is recognized by all criminologists as a part of our sociological evolution towards a more humane and scientific view of punishment'', you can be sure to find nominalizations in there because many big words are formed from nominalizations.
\paragraph{Nominalizations Can Describe Nonsense}
Although nominalizations are useful in conveying in information (e.g. I would not have been able to write this article without nominalizations), they can also mean nothing when used without specifying the verb/adjective they stand for. For example, with nominalizatons, we can form sentences such as ``I love to have more results'' and ``We must choose based on utility'' that are nonsense without knowing what is resulting and what is being used. These cases arise when we use nominalizations in ways that nouns can be used (e.g. as a metaphor) but in ways that do not make sense. It makes sense to ``lose an appetite'' but it does not make sense to ``lose utility'' unless we fully understand what is meant by ``utility''. Nominalizations act as labels that represent thought depending on the contextual sentence, but can be misused when we believe that they themselves hold constant meaning.
\paragraph{Nominalizations Do Not Exist}
Nominalizations do not exist because they are labels for metaphors. You cannot have "poor communication" if you are not communicating right now. More importantly, you can't hold a bag of communication because it doesn't exist in the real world. Just like you can have ``2 apples'' and ``2 swords'', but you cannot have ``2'', you can ``use apples'' and ``use swords'', but you cannot have ``utility'' in the abstract. A critic may say that the label ``2'' does exist, and the critic would be correct. These labels do exist in the sense that they are labels, but they themselves do not exist in the sense that they do not describe anything of the world. Since nominalizations are only labels for metaphors, they do not exist in reality. To recover the precise meaning behind any nominalization, we must look at the verb/adjective it is derived from. If there is no verb/adjective it refers to, then the nominalization is nonsense and should not be used if we are to have a discussion. If there is a verb/adjetive it refers to, then we can continue using the nominalization as long as we know that it is just a label for the real verb/adjective it is formulated from.
\end{document}
