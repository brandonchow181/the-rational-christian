\documentclass[a4paper, 12pt]{article}
\usepackage[T1]{fontenc}
\usepackage[utf8]{inputenc}
\usepackage{babel}
\usepackage[raggedright]{titlesec}
\usepackage{blindtext}
\titleformat{\paragraph}[hang]{\normalfont\normalsize\bfseries}{\theparagraph}{1em}{}
\titlespacing*{\paragraph}{0pt}{3.25ex plus 1ex minus .2ex}{0.5em}
\setcounter{section}{0}
\begin{document}
\section{Errors}
Some Christians hold beliefs about how they should apply scripture that are formed from misinterpretations of scripture. These include the topics of (1) swearing; (2) healthy pursuit of money; (3) sexual lust; (4) masturbation; (5) premarital sex; (6) figurative idolatry; (7) egoic pride. Understanding these seven topics and the scriptural evidence behind them will result in a few things.
\begin{enumerate}
  \item You will protect yourself and others from falling into these errors.
  \item You will understand how the ego fuels these seven errors by playing into conformity culture and the delusion of separation.
  \item You will realize the power of the subconscious mind in creating and removing feelings of guilt.
\end{enumerate}
We can divide these seven errors into two categories: (1) that which is permissible which is viewed as sinful (swearing, sexual lust, masturbation, premarital sex, figurative idolatry); (2) that which is sinful which is viewed as permissible (love of money, egoic pride). \\
\break
While both tasks appear similar, the first is more consuming than the second because while the ego may be shown that a behavior is not sinful, the ego wishes to keep its own desires to conform and enforce separation. Consequently, the ego forms several arguments to show that Christians should not do all that is permissible. To avoid repeating myself for each error, I have organized my responses to all of these arguments below. I suggest the reader come up with more counterexamples to these common arguments as an exercise. \\
\pagebreak
\subsection{Whether Christians should continue doing permissible activities that the predominant Christian culture views as sinful?}
\paragraph{1. Argument from Feeling} \textit{If it feels wrong, it's probably wrong.} \\
Though the laws of God are written on our hearts (Romans 2:14-15), we cannot rely on our feelings as an accurate heuristic to discern what is sinful due to the effect of biological instinct (nature) and social conditioning (nurture). Since both of these heuristics can be wrong, any argument from feeling is inappropriate in our discussions. I will give examples when (1) nature fails and then when (2) nurture fails. \\
\break
(1) According to the argument from feeling, if I feel that it is wrong to say no to a baby when the baby asks to drink a bottle of champagne, then I must be doing something wrong. Although we are biologically predisposed to care about babies, the Bible does not say that it is wrong to deny champagne to a baby. \\
\break
(2) According to the argument from feeling, if I feel that it is wrong to eat meat because I feel bad for the animals, then I must be doing something wrong. However, Jesus says that ``it is not what goes into the mouth that defiles a person, but what comes out of the mouth'' (Matthew 15:11 ESV). Therefore, our feelings about certain actions are often inaccurate our natural instincts do not reflect Biblical truths. While our feelings can determine what is right when it comes to murder or lying, notice that our feelings only prove to be right when our cultural norms match biblical truths. Otherwise, our feelings can tell us nothing. \\
\paragraph{2. Argument from Conformity} \textit{If most people think it's wrong, it's probably wrong.} \\
Although many people in society view that it is inappropriate for people to publically share their religious beliefs, Jesus commanded his disciples to ``go therefore and make disciples of all nations''. The argument from conformity makes the unsupported presupposition that most people know what is right. Since the truths of the Bible can be both unpopular and true, the argument from conformity fails to provide us with the truth. \\
\pagebreak
\paragraph{3. The Idolatry Argument} \textit{Whatever permissible behavior you are doing can become idolatrous if a person is doing it in a way that rebels against God's law.} \\
If a behavior is permissible, it is by definition not against God's law. Therefore, by doing the permissible behavior, we are not treating the behavior as God since we are not deviating from God's established law for what is right and wrong. Since every behavior under the sun can become idolatrous if we worship it, taking the idolatry argument to its logical conclusion means doing absolutely nothing on this Earth since everything can be used against God's law. Therefore, we must separate an behavior's sinfulness from its capacity to be used sinfully if we are to determine what is sin and what is not. \\
\paragraph{4. The Fellowship Argument} \textit{We should not actively say things that will divide the church community even if they are true.} \\
Many of these permissible behaviors that the Christian community looks down upon has already caused much discord within churches. Continuing to spread these misunderstandings of scripture maintains the appearance of harmony but perpetuates the inner conflict that many Christians have when it comes to these permissible behaviors. Given the \underline{truth-good hypothesis}, giving Christians the truth will provide Christians a solid foundation for their behaviors rather than turn them into Pharisees who perpetuate misinterpretations of scripture for their ego's survival. \\
\paragraph{5. The Stumbling Block Argument} \textit{We should not openly say that it is fine to do a permissible behavior because we will then cause other Christians to stumble. I cite an example from Paul where I tells Christians not to eat meat if it causes others to sin.} \\
While it is true that Paul calls Christians not to do anything (even that which is not sinful) if it causes another Christian to stumble, we should also teach as Paul taught about food that we need not bear unnecessary guilt from permissive behaviors. Following Paul's example, if a person is not able to talk about the sinfulness of a behavior without compromising his own values then that person should not be included in the discussion. I will add specific ways to handle the stumbling block argument when I discuss the errors. \\
\pagebreak
\paragraph{6. The Disingenuous Argument} \textit{You're just saying this because you want to give yourself an excuse to sin.} \\
While the disingenuous argument questions the integrity of the person who's proposing that a permissible behavior is not sinful, it does not at all address the sinfulness of the behavior. All the disingenuous argument does is shift the discussion away from what is sinful into a matter of the person's intentions. \\
\paragraph{7. The Meta Argument} \textit{If you have to spend the energy to show that an behavior is permissible, then you're only trying to convince yourself that the behavior is permissible through getting validation.} \\
This is a variation of the disingenuous argument. Other than what I have written there, I will give an example from the Bible. Jesus spent time explaining why eating meat is not a sin in Matthew 15:11. Assuming that Jesus did not sin, we have an example here of a person explaining that eating meat is not sin. However, by the meta argument, we reach the absurd conclusion that Jesus only says Matthew 15:11 to excuse himself of sin. If a questioner keeps diverting the discussion to a particular person's integrity, the integrity of entire discussion is put at risk since the discussion becomes no longer about the sinfulness of the behavior but about the motivation of the person. \\
\paragraph{Final Remarks} What is common to all of these seven arguments is that they divert the discussion away from what is sinful. These arguments all try to deal with the implications of whether a behavior is sinful before dealing with the behavior itself. While these objections may come from a questioner's genuine desire to create harmony in the Christian community, spreading misinterpretations of scripture can only cause deeper damage to the Christian community by making Christians increasingly disillusioned when they understand the truth. These objections distract from the real discussions we should be having based on scriptural evidence. \\
\pagebreak
\paragraph{Troubleshooting} If you face numerous of these ad hominem type objections that divert the discussion away from whether a behavior is sinful or not, then follow the three step process for reframing the discussion. \\
\begin{enumerate}
  \item Reject the frame. Say no to the objection by describing the fallacy.
  \item Assert your own frame. Describe how the objection changes the nature of the discussion and move the discussion back on scriptural evidence.
  \item Qusetion the questioner's motive. Ask the questioner why he/she is running away from he scriptural evidence. Once the questioner understand the fallacy behind his/her objections, return back to the scriptural evidence.
\end{enumerate}
Notice that all three levels of this reframing aims to refocus the discussion back onto the scriptural evidence. \\
\subsection{A New Theory Of Handling Errors}
In place of these objections, I propose a new theory of handling errors within the Christian culture based off the \underline{truth-good hypothesis}. Rather than focus on maintaining the harmony of Christian culture, I say that Christians should focus on determining what is true and let harmony develop as a result of the truth. The principles behind this theory are as follows: \\
\begin{enumerate}
  \item Inform all Christians of the truth.
  \item Allow Christians who find permissive behavior repulsive to continue to refrain from them.
\end{enumerate}
If we do not put the truth at the center of Christian culture, people will use Christians that find permissive behavior repulsive as a reason to discourage such permissive behavior. If we do not clearly differentiate what is sinful and what is permissive, we will allow the Christian fads of the day to determine what people feel is right and what is wrong. \\
\end{document}
