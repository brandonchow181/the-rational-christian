\documentclass[a4paper, 12pt]{article}
\usepackage[T1]{fontenc}
\usepackage[utf8]{inputenc}
\usepackage{babel}
\usepackage[raggedright]{titlesec}
\usepackage{blindtext}
\titleformat{\paragraph}[hang]{\normalfont\normalsize\bfseries}{\theparagraph}{1em}{}
\titlespacing*{\paragraph}{0pt}{3.25ex plus 1ex minus .2ex}{0.5em}
\setcounter{section}{0}
\begin{document}
\section*{Future}
\paragraph{Definition Of The Future}
According to the Oxford Dictionary, the future is ``a period of time following the moment of speaking or writing; time regarded as still to come''.
\paragraph{Nothing In The Future Exists}
Notice that everything that is of the future is an indexical since it can only be made sense if we know what the word ``future'' refers to. On its own, the word ``future'' means nothing. However, notice that even if we say that the future is anything that happens after the present moment, we cannot know what it refers to because whatever is in the future has not happened yet. For example, while the phrase ``my job'' is not a hanging indexical if I currently have a job, the phrase ``my future job'' is a hanging indexical because I cannot have a ``future job''. Consequently, nothing in the future exists. If it did exist, it would be in the present moment and not the future.
\paragraph{The Future Does Not Exist}
The implication of this realization is that the future does not exist. The word ``future'' points to nothing of the world and therefore does not exist just like the ability to fly in cars does not exist (at the current time of this writing). When we talk about something such as the year 3019, we are not actually talking about the year 2050 but our present projection of what the year 3019 will be like. No one can possibly know what the year 3019 will be like since no one can live in the future since by definition, the word ``future'' refers to a time that has not yet happened.
\end{document}
