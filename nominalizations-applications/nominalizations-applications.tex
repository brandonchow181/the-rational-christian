\documentclass[a4paper, 12pt]{article}
\usepackage[T1]{fontenc}
\usepackage[utf8]{inputenc}
\usepackage{babel}
\usepackage[raggedright]{titlesec}
\usepackage{blindtext}
\titleformat{\paragraph}[hang]{\normalfont\normalsize\bfseries}{\theparagraph}{1em}{}
\titlespacing*{\paragraph}{0pt}{3.25ex plus 1ex minus .2ex}{0.5em}
\setcounter{section}{0}
\begin{document}
\section{Nominalizations: Applications}
Given what we know about nominalizations, here are some of its applications when it comes to (1) need; (2) love; (3) anxiety. \\
\subsection{Need}
\paragraph{Formulation Of A Need}
Since ``need'' is a nominalization of the verb ``to need'', we know that ``needs'' do not exist, but instead represent what happens when a person needs to do something. However, it makes no sense to ``need to do something'' without understanding a ``need to do something for something else''. Notice that every thing you need can be formulated in the structure ``I need x to get y''. The following are examples of this structure.
\begin{itemize}
  \item ``I need to continue to eat food to survive.''
  \item ``I need a 3.9 to be admitted to a top ten medical school.''
  \item ``I need to see you to be happy.''
\end{itemize}
\paragraph{No One Needs Anything If It Is Not Needed For Anything}
Without using the structure mentioned in the previous paragraph, we get the following sentences.
\begin{itemize}
  \item ``I need to continue to eat food.''
  \item ``I need a 3.9.''
  \item ``I need to see you.''
\end{itemize}
The last three sentences do not make sense without understanding what the need is for. After all, if a person was undergoing a fast, then the statement ``I need to continue to eat food.'' would be false. Similarly, if a person was trying to dropout of school, then the statement ``I need to a 3.9.'' is false and if a person wanted to be sad, then the statement ``I need to see you.'' would also be false. Although no one wants these culturally ``bad'' things, I use these sentences to make the point that we only have a need if it fulfills a condition for something else. \\
\paragraph{The Infinite Regress of Needs}
Notice that if all needs are set up in the form ``I need  x to get y'', we must reach a y that we do not need, or else we are saying that ``we need y to get z'' and so forth. With this chain of reasoning, there must be something that we are choosing not out of need. We can call this the arbitrary aim. For many people, this arbitrary aim is the ``need to be okay'' since even other possible needs such as the ``need to be happy'' and the ``need to be satisfied'' can be reduced to some form of ``being okay''.
\paragraph{You Are Always Okay Now, But You May Not Always Think You Are Okay}
\paragraph{Surrending To the Moment Releases You Of Needs}
\subsection{Love}
\subsection{Anxiety}
\end{document}
