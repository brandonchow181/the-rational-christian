\documentclass[a4paper, 12pt]{article}
\usepackage[T1]{fontenc}
\usepackage[utf8]{inputenc}
\usepackage{babel}
\usepackage[raggedright]{titlesec}
\usepackage{blindtext}
\titleformat{\paragraph}[hang]{\normalfont\normalsize\bfseries}{\theparagraph}{1em}{}
\titlespacing*{\paragraph}{0pt}{3.25ex plus 1ex minus .2ex}{0.5em}
\setcounter{section}{0}
\begin{document}
\section*{Christian Culture}
It seems that Christian culture can be thought of as the set of common behaviors adopted by the majority of Christians. As Christians congregate, they begin to treat these common behaviors as normal out of the ego's desire to conform to what is acceptable to the most people. Through modern forms of communication such as articles, books, videos, and word of mouth, a predominant set of behaviors begin to manifest as the predominant Christian culture, which often passes down from generation to generation. Since many Christians follow these behaviors in order to behave as a Christian should, the important question we should ask ourselves is whether or not these behaviors are biblical. To answer this question, there appears to be two possible cases: either (1) the predominant Christian culture has interpreted scripture entirely correctly\footnote{\label{mythsfn1}ie. how scripture was intended to be interpreted}} or (2) the predominant Christian culture has not interpreted scripture entirely correctly\footnote[\ref{mythsfn1}] . Someone may criticize me for assuming such a thing as "Christian culture" since so many different "Christian cultures" exist throughout the world with varying behaviors. While it is completely true that within each "Christian culture", we can always find more "Christian subcultures".


Whether or not a Christian has accepted his or her own sexual lusts, that Christian likely feels guilt when lusting after someone else. If such a Christian feels no guilt, then the Christian ignores the predominant church culture either out of a desire to be free from religious authority or out of a
