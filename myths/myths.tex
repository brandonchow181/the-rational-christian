\documentclass[a4paper, parskip=full, 12spt]{article}
\usepackage{parskip}
\usepackage{xparse}
\usepackage{titlesec}
\setcounter{section}{0}
\begin{document}
\section{Myths}



\subsection{Whether Christians should continue doing permissible activities that the predominant Christian culture views as sinful?}
\textbf{Objection 1 (The Feels-Bad Argument)} If it feels bad, it's probably wrong. \\
\textbf{Objection 2 (Argument from Conformity)} If most people think it's wrong, it's probably wrong. \\
\textbf{Objection 3 (The Idolatry Argument)} Whatever permissible activity you are doing can become idolatrous if a person is doing it to rebel against God's law.\\
\textbf{Objection 4 (The Fellowship Argument)} We should not actively say things that will divide the church community. \\
\textbf{Objection 5 (The Stumbling Block Argument)} We should not openly say that it is fine to lust after a man or woman because we will then cause other Christians to stumble. Therefore, whether or not sexual lust for a single person is sinful or not, we should discourage sexual lust as much as possible for the sake of preserving the Christian community. \\
\textbf{Objection 6 (The Disingenuous Argument)} You're just saying this because you want to give yourself an excuse to sin. \\
\textbf{Objection 7 (The Meta Argument)} If you have to spend the energy to show that an activity is permissible, then you're only trying to convince yourself that the activity is permissible through getting validation. \\
\textbf{On the contrary,} these objections are all fallacious and too psychologically biased to be used for any real search for what is sinful or not. \\
\textbf{I respond,} although these objections may come from a genuine desire to create harmony in the Christian community, I believe that the prevalence of these misinterpretations of scripture have caused much damage to the Christian community that can only be alleviated by piercing these misinterpretations with the truth. While everyone has their own interpreation of scripture, these objections are dangerous because they encourage groupthink and are completely fallacious. They obscure the real discussions we should be having based on scriptural evidence and promote conformity instead of truth-seeking behavior. \\
\textbf{Reply to Objection 1 (The Feels-bad Argument)} Even though the laws of God are written on our hearts, we cannot rely on them to help us discern what is right and wrong. Some people believe that it is wrong to meat because they feel it is wrong. Some people believe that it is wrong to openly share one's faith. However, Paul says that it is not wrong to eat meat, showing that our own feelings are not a good indicator of what is right and wrong. And Jesus tells his followers to share their faith. Often times, our feelings about a certain behavior come from some primal mechanism, some observation from our formative years, or from repetitive conditioning to associate a certain behavior with a feeling of disgust. People use the feels-bad heuristic to prey upon a person's already existing conditioning. It is would be useless to tell an unsocialized person to never do what doesn't feel bad because many wrong things may not feel bad to him. Therefore, this heuristic does not belong in any substantive intellectual discussion because of its numerous flaws. \\
\textbf{Reply to Objection 2 (Argument from Conformity)} Many people view that it is wrong for people to publically share their religious beliefs. However, God tells us to be open and share our faith. Therefore, it is fallacy to say that we should not do things that most people in society think are wrong. \\
\textbf{Reply to Objection 3 (The Idolatry Argument)} By definition, if a behavior is permissible, it is not against God's law. Therefore, by doing the activity, we are not "being God" since we are not establishing contrary views of what is right and wrong. Note that whether or not a permissible behavior can become idolatrous is irrelevant here. Any activity, even a popular activity amongst Christian culture, can become idolatry. We have to separate an activity's sinfulness from its capacity to be used as a a sin. \\
\textbf{Reply to Objection 4 (The Fellowship Argument)} Many of these permissible behaviors that the Christian community looks down upon has already caused much discord within churches. There is a strong argument to be made that clarifying what is actually sinful will actually resolve some of the inner turmoil many Christians face during their lives. In doing so, we alleviate the self-inflicted guilt that many Christians have faced and consequently, bring the Christian community closer together. These myths have only served to divide Christians through creating more self-inflicted guilt based on misinterpretations of scripture. \\
\textbf{Reply to Objection 5 (The Stumbling Block Argument)} While it is true that Paul calls Christians not to do anything (even that which is not sinful) if it causes another Christian to stumble, we should also teach as Paul taught about food that it is not sinful for a single man or woman to lust after another person so that we do not bear unnecessary guilt. If we wish to discuss the sinfulness of a certain activity but cannot talk about any issues that could cause someone in the discussion to stumble, then that person should not be in the discussion. \\
\textbf{Reply to Objection 6 (The Disingenuous Argument)} This response doubts the integrity of the person who's proposing that lust isn't sinful instead of dealing with the argument. \\
\textbf{Reply to Objection 7 (The Meta Argument)} Paul spent time explaining why eating meat is not a sin. According to this fallacious argument, Paul was only trying to convince himself that eating meat is not sinful so that he can better indulge in his sin. We know this is not the case so this heuristic is erroneous. \\
\end{document}
