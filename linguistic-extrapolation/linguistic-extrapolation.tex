\documentclass[a4paper, 12pt]{article}
\usepackage[T1]{fontenc}
\usepackage[utf8]{inputenc}
\usepackage{babel}
\usepackage[raggedright]{titlesec}
\usepackage{blindtext}
\titleformat{\paragraph}[hang]{\normalfont\normalsize\bfseries}{\theparagraph}{1em}{}
\titlespacing*{\paragraph}{0pt}{3.25ex plus 1ex minus .2ex}{0.5em}
\setcounter{section}{0}
\begin{document}
\section*{Linguistic Extrapolation}
Linguistic extrapolation occurs when a person uses a word or phrase outside of its common usage. However, when words are extrapolated this way, we can form questions without answers since our language will appear to show that such a question makes sense.
\paragraph{Definition of Extrapolation}
According to the Oxford Dictionary, to expolate something is ``to estimate something or form an opinion about something, using the facts that you have now and that are valid for one situation and supposing that they will be valid for the new one''. When what we know for a particular context ends up being invalid for the context we're looking for, we say that extrapolation fails.
\paragraph{Extrapolations Are Everywhere}
We use extrapolations to form sentences that we have never heard before. For example, suppose that I know that saying the sentence ``can I have a cup of coffee?'' gets me a cup of coffee. Then, given this information, I can use that sentence structure to ask for a puppy by saying ``can I have a puppy?''. In this way, we use word or phrases that we know how to use and apply them in new situations assuming they will function the same way.
\subsection{Extrapolating To Nonsense}
However, even though extrapolating sentence structure is vital for us to form new sentences, extrapolation can fail just like it can fail in mathematics. For example, in mathematics, if I know that f(x)=1/(x-1) is a function, I may assume that this function can return an f(x) value for every x value I substitute in the function. However, substituting x=1 in the function gives an undefined result. Similarly, when we use a particular word or phrase so often in everyday usage, we may believe that such a word actually describes a tangical thing and therefore has its own ``function'' in some sense, with its own undefined points. I will illustrate this in the next examples.
\paragraph{Why am I here?}
\paragraph{What do I want out of life?}


The way we first under
\end{document}
