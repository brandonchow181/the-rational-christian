\documentclass[a4paper, 12pt]{article}
\usepackage[T1]{fontenc}
\usepackage[utf8]{inputenc}
\usepackage{babel}
\usepackage[raggedright]{titlesec}
\usepackage{blindtext}
\titleformat{\paragraph}[hang]{\normalfont\normalsize\bfseries}{\theparagraph}{1em}{}
\titlespacing*{\paragraph}{0pt}{3.25ex plus 1ex minus .2ex}{0.5em}
\setcounter{section}{0}
\begin{document}
\section{Programming}
\subsection{Theory}
Have you ever held a belief that you don't know where you got? That belief is part of your body's programming within your neurology. Since I will continue to use the programming metaphor in other sections, I dedicate this section to explain everything about our programming so that we can use the metaphor in the future while knowing what it stands for.
\paragraph{The Formulation Of Our Programming}
When our bodies developed from the ages of 0 to 7, they were quick to absorb the behaviors of the people around them. We label this process as internalization because it involves a person's neurology adopting something observed outside and copying the behavior into its internal neurology. As children, our bodies needed to internalize and mimic these behaviors so that they could arise out of childhood into adulthood ready to handle the world and it's realities. For this reason, our bodies absorbed the behaviors that the people around us used to solve problems (ie. to get what they want). For example, if a child wants something, the child learns to say "can I have x please?" to get what the child wants. This solution to a perceived problem is called a program because like a computer program it has arisen as a way to handle a particular issue, taking in an input (an event internally or in the environemnt) and an output (a response to resolve the issue). Although we say that these behaviors are programs, they represent real neurons within our brain that store neurological pathways to record when certain behaviors prove to be useful. For the most part, these programs are vital for our survival. Without them, we would come into adulthood having to hammer through repetition everything we needed to know, including how to deal with our own failures, deal with other people, and deal with the environment. Once we come out of a lower level of consciousness, however, we begin to rely on these programs as the people around us begin to treat us as responsible human beings. Through this process, we rely more readily on our programs to support us when we are in perceived trouble. The more we rely on these starting programs, the more these programs (ie. neurological pathways) become strengthened because they become conditioned to activate whenever we are in the environments we inhabit. Regarding this, there are two rules: whatever happens is strengthened, whatever does not happen is weakened. The biological version of these rules are: neurons that fire together wire together, neurons that fire apart wire apart. These rules arise from the fact that when we perform a certain behavior in an environment over and over, that behavior becomes the predominant behavior we use in that environment. This behavior is known as a habit. For this reason, whatever behaviors we absorbed during our body's formative years manifest themselves more powerfully when our bodies enter adulthood. It is no wonder that people often appear like their parents in their mannerisms because people had no one else to absorb such behaviors from. Although these programs do not determine a person's entire life outcome, they determine the default solutions people have to the problems around them. The sum of all the internalized behaviors a person has accumulated throughout their childhood is known as their \textbf{initial programming}. \\
\paragraph{The Formulation Of Difficulties}
However, during our childhood, it is often not possible to absorb behaviors to every possible issue a person may encounter. If a person who has arisen to a full state of consciousness faces an issue that the person has no program for, the person will face the issue with extreme difficulty because the person's programming does not support the issue. For example, a person who has never tied a rope will have extreme difficulty tying a rope in adulthood. And a person who has never played an instrument will have extreme difficulty learning the violin in adulthood. All recurring hardship arises out of a lack of sufficient programming to handle a certain situation. Therefore, if a person faces a recurring hardship in getting an erection during sex, this means that the person's programming does not support the engage of sexual activity. \\
\paragraph{Resolving Difficulties: Lack Of Programming}
If a difficulty arises out of a lack of sufficient programming, then the common way to overcome the difficulty is through repetition. Following the rule we established earlier that that which happens is strengthened, a person can improve at a certain behavior by repeating that behavior over and over. For example, since we were never taught how to drive a car during our childhood, we had to learn how to drive a car through repetition. \\
\paragraph{Resolving Difficulties: Counter-Programming}
If a difficulty arises out of counter-programming, then the person has to either repeat the new program over and over to strengthen it, or if this is not possible, to first remove the problematic programs in order to overcome the difficulty. For example, suppose that a person faces the difficulty of always saying yes to people. Then, in order to resolve the difficulty the person has to either keep saying no over and over to people or to disidentify with the problematic program. In many of these cases, the person has to be convinced that the initial program is outdated or else the body will naturally keep diverting to the initial program out of habit. \\
\paragraph{Understanding Programs}
Much of the difficulties that arise in a person's life situation come from the body's inability to adapt well after childhood. This occurs because the function of a person's programming is purely to survive and to grow (in a way that person wants to grow). Once a person reaches enough means to survive, the person's programming no longer readily relearns whatever it has used to survive. A person's programming is resilient to change because (1) it has been strengthened through repetition and (2) the body relies on the programming to survive. Since the body uses programs to survive, the initial programs the body has internalized become more and more ``default'' as time goes on. For this reason, \textbf{the easiest moment to change a person's programming is today}. Even though the mind may delude itself into thinking that change can happen at any time, it is still grounded by its programming. \\
\paragraph{Moving From Behaviors To Beliefs}
While behaviors deal with ways to handle issues, beliefs deal with what is real. I talked about behaviors first, because these are the foundational building blocks upon which the other material that we absorbed is based upon. However, our neurology does not only absorb behaviors, but also uses behaviors to form beliefs. The process through which the neurology forms beliefs is through presuppositions. Whatever the body senses that a behavior presupposes something to be true of the world, a belief is formed. We can create these beliefs through the following formula: "people only behave this way because x is true". For example, if a person observes people running away from a fire, the neurology forms a belief that fire must be something bad (ie. something to run away from). Notice that the neurology never absorbs what is being said most of the time, but what is being presupposed from what is said. For example, if a person observes his parents telling him to eat vegetables first in order to get some free time, the person will form the behavior of eating vegetables first, but also the belief that vegetables are bad (ie. they are an obstacle to his goals). The fact that what is observed can be rejected but what is implied can rarely be gives way to the general rule that people can reject what we tell them but not what they conclude themselves (or what their neurology assumes to be true given a certain behavior). These beliefs that a person forms during his childhood together form a person's initial beliefs. All of the rules that applied to behaviors also apply to beliefs. The more times a person relies on a belief to function in the world, the more that the belief is strengthened (ie. it becomes integral to a person's neurology). The more times a person can live perfectly fine without a belief to function in the world, the more that the belief is weakened (ie. it becomes less integral to a person's neurology). Furthermore, difficulties also arise when a person has to make a choice without knowing what to believe. I will discuss those difficulties in the next paragraph. \\
\paragraph{Resolving Difficulties: Lack Of Beliefs}
If a person faces difficulties due to a lack of beliefs, all the person has to do is to use his mind to formulate a belief based on everything else that the person believes is true. This can be done through reasoning, through researching evidence, communicating with others etc. \\
\paragraph{Resolving difficulties: Limiting Beliefs}
If a person holds a belief that counteracts what a person wants to do, we say that the person has a limiting belief. Most often limiting beliefs are not formed consciously, but out of the behaviors that a person has observed during his chidhood. These limiting beliefs can be removed through specifically doing things to show the body that its old belief was not comprehensive. \\
\subsection{Freeing Yourself From Programming}
\paragraph{Disidentifying From Programming}
Even though our body's neurological programming affects our life situation, including our personalities, our skills, and our preferences, we have to remember that our programming is not us. The moment we say that we are the programming, we cannot free ourselves from it. We can realize that we are not the programs within us by realizing that we would still experience reality even if we lost a specific behavior. Even if we lost all of our programs and became imbeciles, we would still exist as that which undergoes experience. Knowing this, understand that if we face a difficulty, it is nothing to do with us, but likely our programming if it is not related to anything biologically hardwired such as a fear of heights or small insects. \\
\paragraph{Recognizing Programs That Further Our Goals}
The majority of the programs within us help us achieve our goals, such as our instinct to look both ways before crossing a street, or saying ``thank you'' after someone does something that helps further our own goals. Without these programs, we would have to be fully conscious to remember to look both ways when crossing a street or try to memorize what to say after someone does something to further our own goals. \\
\paragraph{Recognizing Programs That Blocks Our Goals}
While most of our programs further our goals, there are programs within us that do not further our goals. For example, suppose that a person was once bit by a dog. That person may have developed the program to run away from dogs when the person sees one. However, the person's neurology no longer has realized that the threat has long been gone and that the program protects the person from nothing. Even if we include the threat of being bit by another dog, suppose for example that the person believes that the threat of dogs is far too small to justify a constant anxiety of seeing dogs. This would be an example of a person's neurology adopting programs that blocks a person's goal in life. \\
\paragraph{Changing The Mindset}
As mentioned before, a person's neurology keeps programs because they . The key to freeing ourselves from certain programs is to retrain the mind that it does not need these programs to function properly.
\paragraph{Reframing}
\subsection{Case Study}
Here are some behaviors that I have observed within my own life that have formed some of the programming within my own neurology.
\begin{enumerate}
  \item I cannot recall the first year (and most of the second year) of my life's situation because at those ages my hippocampus was not fully formed.
  \item When I was two years old, I remember celebrating my third birthday.
  \begin{itemize}
    \item The neurology internalized that birthdays are important and that we should celebrate other people's birthdays.
  \end{itemize}
  \item When I was three years old, my family used another family's swimming pool when the other family was away.
  \item When I was four years old, I saw a commercial for medication for depression. When I told my mom that I felt like the person in the advertisement, my mom did not give any reaction, causing me to conclude that feeling that feeling like the person in the commercial was natural and normal.
  \item When I was five years old, one of my friends did not want me to enter a playhouse with him because his area (that he made up) was for people who were six years old.
  \item When I was six years old, I got in trouble for sitting on a chair while rocking it such that the back two legs were off the ground even though I was repeated told not to.
  \item When I was seven years old and away in Hong Kong for vacation, I was not able to finish my homework causing my mom to get very upset at me.
  \item When I was eight years old, I got into a heated argument with someone over a physical game. I pinched the other person and got into trouble. After that, my parents got upset at me and the person I pinched later showed signs of aggression towards me for the next few years with the matter being unresolved.
  \item When I was nine years old, I played a game called Bloons Tower Defense on my home computer. After a few days, I noticed that that particular website was blocked even though other websites were not blocked. Although I expected my parents had done something to the computer, I looked to my parents only to find that they did not appear to have changed in their behavior.
  \item When I was ten years old, one of my friends that I frequently hung out with stopped hanging out with me.
  \item When I was eleven years old, my parents were very awkward with me about discussing whether I liked girls.
  \item When I was twelve years old, I was at an event with friends. At the end of the event, one of my friends invited everyone else except for me to watch the Avengers. Having already assumed that it would be "needy" to invite myself somewhere, I did not see the movie with them.
  \item When I was thirteen years old, my parents were upset that I was not investing enough time to finish a speech script for speech and debate.
  \item When I was fourteen years old, I wanted my parents to drive me to meet a girl. However, before letting me leave, my mom made me write down the person's name and phone number on a slip of paper to put in an envelope in case I was in any danger.
  \item When I was fifteen years old, my parents caught me using my phone to text people after I went to bed, making me feel that I did something wrong by not letting them know beforehand.
  \item When I was sixteen years old, one of my friends said it was inappropriate for me to want to hang out with her. Even though my feelings were hurt, the person never answered any of my messages afterwards causing me to lose a sense of closure.
  \item When I was seventeen years old, I went to get a snack with some friends late at night. Although I had a great time, I came home later than my parents expected, causing them to bring up past times when I came home later than expected.
  \item When I was eighteen years old, my parents were upset at me for not putting a math placement exam as a high priority before other schoolwork.
  \item When I was nineteen years old, my parents were upset at me for making a chat group that could damage their reputation.
  \begin{itemize}
    \item The neurology internalized that a reputation is real and that we should do whatever to keep it to survive.
  \end{itemize}
  \item When I was twenty years old, my parents were upset that I did not stay at home enough and that I was believing in things that were too different from normal people.
\end{enumerate}
\subsection{Raising Children}
\paragraph{Principles}
\paragraph{The Parable Of The Talents}
``He also who had received the one talent came forward, saying, ‘Master, I knew you to be a hard man, reaping where you did not sow, and gathering where you scattered no seed, so I was afraid, and I went and hid your talent in the ground. Here, you have what is yours.’ But his master answered him, ‘You wicked and slothful servant! You knew that I reap where I have not sown and gather where I scattered no seed? Then you ought to have invested my money with the bankers, and at my coming I should have received what was my own with interest. So take the talent from him and give it to him who has the ten talents.'' (Matthew 25:24-28)

\end{document}
