\documentclass[a4paper, 12pt]{article}
\usepackage[T1]{fontenc}
\usepackage[utf8]{inputenc}
\usepackage{babel}
\usepackage[raggedright]{titlesec}
\usepackage{blindtext}
\titleformat{\paragraph}[hang]{\normalfont\normalsize\bfseries}{\theparagraph}{1em}{}
\titlespacing*{\paragraph}{0pt}{3.25ex plus 1ex minus .2ex}{0.5em}
\setcounter{section}{0}
\begin{document}
\section{On Time}
\subsection{The Illusion Of Time}
\paragraph{The Nature Of Time}
Although time appears to be a fundamental part of our experience, it is the one concept that can stop us from accessing the present moment (ie. the reality in all things). We only have to look in our movies, books, and stories, to find that the concept of time is problematic for many characters, who cannot deal with their past memories, their future aspirations, or the passage of time itself. \\
\paragraph{The Formulation Of Time}
When talking about the nature of time, we cannot refer to it in terms of itself if we want to explore what time is. For example, if I were to explain to someone that a minute is sixty seconds, I am not clarifying what time is. And if I were to explain to someone that a second is the time it takes for the second hand to move from one tick mark to another, I would be using time to define time. Our initial task, then, is to define time in way that does not involve time itself. If we cannot do this, then we will say that time is a fundamental quantity like distance that cannot be explained. \\
\break
To clarify what we are talking about, according to the Oxford Dictionary, time is ``the indefinite continued progress of existence and events in the past, present, and future regarded as a whole''. For now, I am not dealing with the other definition of time as ``a point of time as measured in hours and minutes past midnight or noon''. An example of the first definition is ``half an hour'' and an example of the second definition is ``3:30pm''. Although both of these definitions of time are different but closely related, we will first examine the first definition first. \\
\breaks
I will first propose that \textbf{time is a measurement} just like distance is a measurement. When the second hand of a clock ticks once (ie. it moves from one mark to another mark), we say that one second has passed even though all that has happened is that a long tiny stick has moved from one position in 3D space to another position in 3D space. Similarly, the International System of Units (SI) defines the second as 9,192,631,770 oscillations of a caesium-133 atom. In both of these cases, whether it is the movement of a wooden stick or the oscillation of an atom, we can see that \textbf{time is a measurement of the change of events}. \\
\break
We can check this result through imagining a universe with no changes. In such a universe, nothing can happen because no events take place. Every particle in this universe is perfectly still. Whether we say that time is still passing or that time is frozen does not matter because in such a universe, we can not say whether something is happening before or after something else. For this reason, time only exists as a relevant concept when there are changes to measure in 3D space just like distance only exists as a relevant concept when we are in 3D space. \textbf{Time measures change}. Even though a vacuum may not change at all within a period of ``five seconds'', the concept of ``five seconds'' still exists because something else is changing in the universe whether it is the atomic clock in Washington D.C. or the grandfather clock in your parent's living room. \\
\paragraph{Time As A Derived Concept}
The main question now is whether or not time measures change or whether change measures time. In other words, we wish to discover whether or not (1) we have made up an abstract concept called time to keep track of changes or whether or not (2) the changes we observe in the universe reflect an invisible force called time that is a feature of our universe. Of the first type, we have numbers, which can measure the apples I have even though numbers do not exist in the universe (ie. there is no part of 3D space where I can find ``2'' in the abstract). Of the second type, we have light sensors, which can measure the number of photons in a given space in a certain moment. Settling this question determines which is the derived concept, change or time. \\
\break
Suppose that time is not a derived concept. Then, changes such as the ticking of a clock measure something real called time. Then, we can say thta we are all moving forward in time since time is always passing. To say ``in three hours'' means to travel in time a certain amount. Furthermore, it is conceivable for everything to stop moving if time were to stop. However, the opposite is not true, since if everything were to stop moving, time could continue to pass. Although these adjustments in how we see time do not lead to anything absurd, the following two examples reveal absurd conclusions. (1) What then does it mean for a time traveler to move backwards in time? Convention would tell us that it means to move backwards through a set of 3D space. In other words, the time traveler would see everything that transpired in reverse. If a glass was shattered, the time traveler would be able to see the shattered glass come back together again. However, the peculiar part of all this is that the time traveler would also be moving forwards in time. Time is moving forwards from the time traveler's point of view because even though everything is happening in a backwards sequence to the traveler, events are still unfolding from his point of view meaning that a progression is happening in 3D space. In fact, if everyone in the world began to move backwards in time, no one would know because while shatterd glasses restoring themselves may appear incredible, it would feel like events are progressing and therefore time is passing in the forward direction. Therefore, since it is absurd to talk about moving backwards in time, it is absurd to talk about moving forwards in time. (2) What would it mean for time to speed up? Convention would tell us that it means that clocks would move much faster and objects would fall at a faster rate. But to us, time would not be passing by at a faster rate if our perception of time also doubled. Therefore, it is meaningless to talk about the passage of time if the rate that time moves is irrelevant. Both of these examples show problems that arise when we treat time as a feature of the universe, a dimension so to speak. These problems arise because we only understand time as a sequence of events. If we keep the idea of sequences of events, we can live without the concept of time. However, if we keep the idea of time, we cannot live without the idea of sequences of events without reaching paradoxical conclusions. \\
\break
To a child, the idea of time as a derived concept is simple to understand. Things happen. Some things cause other things to happen. The first things are called what happens before and the second things are called what happens after. When a number of events have occurred, we say that one second has occurred. However, our conception of time still depends on observed changes to occur. \\
\paragraph{Time Does Not Exist}
Since time is a label we give to a sequence of events, it itself does not exist (ie. it itself is nothing in reality although it refers to something that is in reality). Simply because time does not exist (ie. it is not anything in our 3D universe) does not mean we cannot use time in any discussion. We can use the concept of time as long as we always know what it stands for just like any metaphor. \\
\paragraph{The Illusion Of Time}
Although time does not exist, we have perceived that time exists for so long because we have used the metaphor of time without understanding what the concept of time is derived from. Take a look at the following sentences: (1) ``I need more time to finish the project.'' (2) ``What did you do with your time?'' (3) ``I wasted all of my time.'' (4) ``Where did the time go?'' (5) ``Can I borrow some of your time?''. All of these sentences mean something but none of them have to do with a real thing called time. I will rephrase all of these sentences without the metaphor of time: (1) I need more events to happen before the project can be finished.'' (2) ``What events have just happened?'' (3) ``The events that happened are not the ones that I wanted to have happened. And that is my fault.'' (4) ``I did not realize that so many events happened.'' (5) ``Can you help do some events that I want to have happen?''. These new sentences reveal that when we are talking about ``time'', we are not talking about a real thing. Even though we can live without the concept of time, the concept is still useful to get information across quickly. Refer to \underline{talking about non-existence} for more information on this. \\
\paragraph{Moments}
When it comes to talking about moments such as ``3:30pm on 1/2/2019'', we have to understand that these are labels that describe 3D states of the world. Even though we call ``3 seconds'' and ``3:30pm'' as both ``times'', they are different things and should be thought of such for any productive discussion. While ``3 seconds'' refers to a sequence of events, ``3:30pm'' refers to a state of the world. Both are labels but they refer to different things. \\
\paragraph{Troubleshooting}
If you still cannot understand the sense in which I say that time does not exist, think about the existence of cold. While it may appear that such a thing as ``cold'' is real, we know that it is not because it is simply a label we give for an object we feel is of a low temperature. Suppose that ``cold'' was a real thing. Then, if I touched something and said it was cold, then someone else who touched it would have to say it is ``cold''. However, we know that what is ``cold'' to someone may appear as ``hot'' to someone else. Although the velocity of particles is real, the existence of a real attribute called ``cold'' is a derived notion based on some judgment that our own body puts on the sensation. Furthermore, we can talk about the velocity of particles without talking about ``hot'' or ``cold'', but we cannot talk about ``hot'' or ``cold'' without talking about the velocity of particles. We can only talk of making an object hotter or colder in the sense of raising or lowering the velocity of the particles. (Note that the sensation of hot and cold does exist, even though a real thing called cold or hot does not exist.) \\
\paragraph{Summary}
What we know now is that: \\
\begin{enumerate}
  \item Change (ie. differences in 3D states of the world) is a fundamental part of the universe.
  \item Time is nothing more and nothing less than a measurement of these changes. Without any changes to observe, time is an irrelevant concept.
  \item Time is not a real ``force'' that causes things to move. Time is only a label we give to things that move.
\end{enumerate}
\subsection{Objections}
\paragraph{Objection To Time As A Derived Concept}
The immediate objection I want to discuss is that time is fundamental for so much of our language to make sense. I will list some of these objection below even though I am sure the reader can come up with more. \\
\begin{enumerate}
  \item In our langauge, we talk about before and after. However, how can we talk about before and after if time itself does not exist?
  \item In our language, we use present participles to describe things happening. Interestingly, ``happening'' is itself a present participle. How can we use these words if there is no time for things to happen?
  \item If there is no such thing as time, how is movement possible? How can light travel without time?
  \item If there is no such thing as time, why is what happened ``today'' considered ``yesterday'' tomorrow?
\end{enumerate}
These examples are make the false assumption that if we can use a concept to describe something that's happening, then that concept exists. When we think of time, we have to think of it as a a label like a mask on a person. It itself is not real except for being a label, even though it stands for something real (a sequence of 3D states of the world). For example, a person making one of these objections is like someone who says that because I can use the number 2 to describe 2 apples, the number ``2'' is real. Simply because I can see ``2 apples'' in front of me, does not mean that the apples contain some invisible quality called ``2''. ``2'' is merely a description we give of what is going on. Given this, I will answer the four questions. \\
\begin{enumerate}
  \item We can talk about before/after without time because it is a real concept that can be observed. If someone knocks over a glass and it shatters, I can see that the glass was not shattered and now is shattered. A situation like this does not need ``time'' to occur. It simply occurs. We as humans only apply the label of ``time'' afterwards to summarize what has happened (ie. it took five seconds for the glass to shatter).
  \item We can use present participles without time because we can understand events happening without introducing the concept of time. We only say that ``time passes'' when we see an event happening. However, the event does not need something called ``time'' to happen. ``Time'' is the label we give to things that happen.
  \item Light does not need time to travel because it needs nothing. Photons move through the air not because there is a universal clock ticking or because there is a quantity called time that moves. Photons move through the air because of something that caused them to move.
  \item We say what happened ``today'' as ``yesterday'' tomorrow because it appears before the current state of the world. We do not need a concept of time to exist for us to understand sequence.
\end{enumerate}
The problem with all of these objections is that instead of moving downwards from labels to the existence of physical realities, these objections represent a way of thinking upwards from the existence of physical realities to the existence of labels that we use to describe physical realities. The people making these sort of objections see what is happening, and then infer that some invisible substance called ``time'' must exist for these things to happen when ``time'' is simply a way we easily describe the sequence of events in a quick manner. \\
Another objection is that
\paragraph{Apparant Contradictions}
Someone may say that time must existence for other quantites such as speed to exist. However, we can easily talk about speed, not as distance per time, but as distance per sequence of particular events. As long as we know that time is a label for a sequence of events, we can freely use the word ``time'' to simplify our discussions. \\
\end{document}
